
%% bare_jrnl.tex
%% V1.3
%% 2007/01/11
%% by Michael Shell
%% see http://www.michaelshell.org/
%% for current contact information.
%%
%% This is a skeleton file demonstrating the use of IEEEtran.cls
%% (requires IEEEtran.cls version 1.7 or later) with an IEEE journal paper.
%%
%% Support sites:
%% http://www.michaelshell.org/tex/ieeetran/
%% http://www.ctan.org/tex-archive/macros/latex/contrib/IEEEtran/
%% and
%% http://www.ieee.org/



% *** Authors should verify (and, if needed, correct) their LaTeX system  ***
% *** with the testflow diagnostic prior to trusting their LaTeX platform ***
% *** with production work. IEEE's font choices can trigger bugs that do  ***
% *** not appear when using other class files.                            ***
% The testflow support page is at:
% http://www.michaelshell.org/tex/testflow/



\documentclass[journal]{IEEEtran}
\usepackage{blindtext}
\usepackage{graphicx}
\usepackage{url}


% *** GRAPHICS RELATED PACKAGES ***
%
\ifCLASSINFOpdf
  % \usepackage[pdftex]{graphicx}
  % declare the path(s) where your graphic files are
  % \graphicspath{{../pdf/}{../jpeg/}}
  % and their extensions so you won't have to specify these with
  % every instance of \includegraphics
  % \DeclareGraphicsExtensions{.pdf,.jpeg,.png}
\else
  % or other class option (dvipsone, dvipdf, if not using dvips). graphicx
  % will default to the driver specified in the system graphics.cfg if no
  % driver is specified.
  % \usepackage[dvips]{graphicx}
  % declare the path(s) where your graphic files are
  % \graphicspath{{../eps/}}
  % and their extensions so you won't have to specify these with
  % every instance of \includegraphics
  % \DeclareGraphicsExtensions{.eps}
\fi
% graphicx was written by David Carlisle and Sebastian Rahtz. It is
% required if you want graphics, photos, etc. graphicx.sty is already
% installed on most LaTeX systems. The latest version and documentation can
% be obtained at: 
% http://www.ctan.org/tex-archive/macros/latex/required/graphics/
% Another good source of documentation is "Using Imported Graphics in
% LaTeX2e" by Keith Reckdahl which can be found as epslatex.ps or
% epslatex.pdf at: http://www.ctan.org/tex-archive/info/
%
% latex, and pdflatex in dvi mode, support graphics in encapsulated
% postscript (.eps) format. pdflatex in pdf mode supports graphics
% in .pdf, .jpeg, .png and .mps (metapost) formats. Users should ensure
% that all non-photo figures use a vector format (.eps, .pdf, .mps) and
% not a bitmapped formats (.jpeg, .png). IEEE frowns on bitmapped formats
% which can result in "jaggedy"/blurry rendering of lines and letters as
% well as large increases in file sizes.
%
% You can find documentation about the pdfTeX application at:
% http://www.tug.org/applications/pdftex





% correct bad hyphenation here
\hyphenation{op-tical net-works semi-conduc-tor}


\begin{document}
%
% paper title
% can use linebreaks \\ within to get better formatting as desired
\title{Strange teachers you have here}
%
%
% author names and IEEE memberships
% note positions of commas and nonbreaking spaces ( ~ ) LaTeX will not break
% a structure at a ~ so this keeps an author's name from being broken across
% two lines.
% use \thanks{} to gain access to the first footnote area
% a separate \thanks must be used for each paragraph as LaTeX2e's \thanks
% was not built to handle multiple paragraphs
%

\author{
\IEEEauthorblockN{Marc Juchli} \\
\IEEEauthorblockA{EEMCS,  
Delft University of Technology\\
m.b.juchli@student.tudelft.nl}   %<------ Line breaks in the current column
}

% <-this % stops a space


% note the % following the last \IEEEmembership and also \thanks - 
% these prevent an unwanted space from occurring between the last author name
% and the end of the author line. i.e., if you had this:
% 
% \author{....lastname \thanks{...} \thanks{...} }
%                     ^------------^------------^----Do not want these spaces!
%
% a space would be appended to the last name and could cause every name on that
% line to be shifted left slightly. This is one of those "LaTeX things". For
% instance, "\textbf{A} \textbf{B}" will typeset as "A B" not "AB". To get
% "AB" then you have to do: "\textbf{A}\textbf{B}"
% \thanks is no different in this regard, so shield the last } of each \thanks
% that ends a line with a % and do not let a space in before the next \thanks.
% Spaces after \IEEEmembership other than the last one are OK (and needed) as
% you are supposed to have spaces between the names. For what it is worth,
% this is a minor point as most people would not even notice if the said evil
% space somehow managed to creep in.



% The paper headers
\markboth{The Culture Clash, September~2016}%
{}
% The only time the second header will appear is for the odd numbered pages
% after the title page when using the twoside option.
% 
% *** Note that you probably will NOT want to include the author's ***
% *** name in the headers of peer review papers.                   ***
% You can use \ifCLASSOPTIONpeerreview for conditional compilation here if
% you desire.




% If you want to put a publisher's ID mark on the page you can do it like
% this:
%\IEEEpubid{0000--0000/00\$00.00~\copyright~2007 IEEE}
% Remember, if you use this you must call \IEEEpubidadjcol in the second
% column for its text to clear the IEEEpubid mark.



% use for special paper notices
%\IEEEspecialpapernotice{(Invited Paper)}




% make the title area
\maketitle


\begin{abstract}
%\boldmath

Academia has become very uniform in its conceptual structure. 
However, differences in their teaching methods between different countries still remain due to culture. 
This paper illustrates the differences and similarities between teaching abroad and in the Netherlands. 
A set of key characteristics of teaching will be identified and compared. 
Using video recorded lectures we will compare the implementation of those characteristics in the USA, China and the Netherlands.
In order to improve the sample, we take multiple video lectures of different institutes and lecturers of the same country into account.
The comparison uncovers certain cultural differences in teaching which can be useful for any person travelling from or to the Netherlands for academic purposes.

\end{abstract}
% IEEEtran.cls defaults to using nonbold math in the Abstract.
% This preserves the distinction between vectors and scalars. However,
% if the journal you are submitting to favors bold math in the abstract,
% then you can use LaTeX's standard command \boldmath at the very start
% of the abstract to achieve this. Many IEEE journals frown on math
% in the abstract anyway.

% Note that keywords are not normally used for peerreview papers.
\begin{IEEEkeywords}
technical writing, teaching, culture.
\end{IEEEkeywords}


% For peer review papers, you can put extra information on the cover
% page as needed:
% \ifCLASSOPTIONpeerreview
% \begin{center} \bfseries EDICS Category: 3-BBND \end{center}
% \fi
%
% For peerreview papers, this IEEEtran command inserts a page break and
% creates the second title. It will be ignored for other modes.
\IEEEpeerreviewmaketitle



\section{Introduction}

Studying abroad can not only bring personal difficulties but academic hurdles may be faced as well.
The differences in the way teachers teach varies from country to country.
As a matter of fact, this may lower the ability to learn, simply because a student is not used to this way of learning.
However, if one is aware of those differences and similarities in advance, the learning curve might become less steady.

The purpose of this report is to support students with their learning process while studying abroad. 
Therefore, we use recorded lectures to compare the teaching behaviour in the countries USA, China and the Netherlands.
Due to the fact that only a subset of countries are compared with each other in this paper, the results presented are significant to Dutch students travelling to China or the USA, or Chinese and American students travelling to the Netherlands, exclusively. 

In Section \ref{characteristics} we explain the key characteristics of teaching in general. Section \ref{similarities} provides insight into the correlation between universities in different countries. A brief conclusion in Section \ref{conclusion} highlights what foreign students should be aware of when going abroad.

\section{Key characteristics}
\label{characteristics}

Based on a scheme originally developed by M. Hildebrand, 1971 \cite{hildebrand1971evaluating}, the following list indicates the key characteristics of efficient teaching. Keywords shall provide an understanding of what is meant by each characteristic.

\begin{LaTeXdescription}%[\IEEEsetlabelwidth{Very long label}\IEEEusemathlabelsep]
  \item[(C1) Organization and Clarity] explains clearly; is well prepared; makes difficult topics easy to understand.
  
  \item[(C2) Analytic/Synthetic Approach] contrasts the implications of various theories; gives the student a sense of the field, its past, present, and future directions, the origins of ideas and concepts; discusses viewpoints other than his/her own.
  
 \item[(C3) Dynamism and Enthusiasm] is an energetic, dynamic person; seems to enjoy teaching; conveys a love of the field; has an aura of self-confidence.

 \item[(C4) Instructor-Group Interaction] can stimulate, direct, and pace interaction with the class; encourages independent thought and accepts criticism; uses wit and humor effective; is concerned about the quality of his/her teaching.

 \item[(C5) Instructor-Individual Student Interaction] is seen by students as approachable and a valuable source of advice even on matters not directly related to the course.
  
\end{LaTeXdescription}

\section{Similarities and Differences}
\label{similarities}

Each recording\cite{mit}\cite{harvard}\cite{hongkong}\cite{peking}\cite{delft}\cite{amsterdam} is an example of an introduction course in a technical field at the related university.
This shall provide insight into the correlation between the given university and each key correlation mentioned in Section \ref{characteristics}. 

\begin{table}[!ht]
\centering
\caption{Comparison Universities}
\label{universities}
\begin{tabular}{lllllll}
\textbf{}   & MIT & Harvard & HKUST & PKU & TUDelft & UVA \\
\textbf{C1} & 5   & 5       & 4     & 3   & 4        & 4       \\
\textbf{C2} & 4   & 4       & 2     & 3   & 5        & 4       \\
\textbf{C3} & 4   & 5       & 3     & 3   & 5        & 5       \\
\textbf{C4} & 3   & 3       & 3     & 3   & 4        & 4       \\
\textbf{C5} & 2   & 2       & 2     & 1   & 4        & 4      
\end{tabular}
\end{table}

The correlation range reaches from one to five, where one is no correlation and 5 is total correlation.
The summarized results of a poll with 100 contributors is listed in table \ref{universities}.
It clearly shows that people claim that universities located in the same country have similarities in their teaching characteristics.


\section{Conclusion}
\label{conclusion}

In addition to the comparison of the universities in Section \ref{similarities}, we can extract the participating countries from table \ref{universities}. 
As a result, table \ref{countries} explains the correlation of teaching characteristics to each of the listed countries.
It is noticeable that the Netherlands highly correlate with the USA, but not as much with China.
Consequently, students travelling from the Netherlands to China, or vice versa, must be aware of very different teaching style.


% Please add the following required packages to your document preamble:
% \usepackage[table,xcdraw]{xcolor}
% If you use beamer only pass "xcolor=table" option, i.e. \documentclass[xcolor=table]{beamer}
\begin{table}[!ht]
\centering
\caption{Comparison Countries}
\label{countries}
\begin{tabular}{
>{\columncolor[HTML]{FFFFFF}}l 
>{\columncolor[HTML]{FFFFFF}}l 
>{\columncolor[HTML]{FFFFFF}}l 
>{\columncolor[HTML]{FFFFFF}}l }
\textbf{}   & \textbf{USA} & \textbf{China} & \textbf{Netherlands} \\
\textbf{C1} & 10           & 7              & 8           \\
\textbf{C2} & 7            & 5              & 9           \\
\textbf{C3} & 9            & 6              & 10          \\
\textbf{C4} & 6            & 6              & 8           \\
\textbf{C5} & 4            & 3              & 8          
\end{tabular}
\end{table}


% Can use something like this to put references on a page
% by themselves when using endfloat and the captionsoff option.
\ifCLASSOPTIONcaptionsoff
  \newpage
\fi



% trigger a \newpage just before the given reference
% number - used to balance the columns on the last page
% adjust value as needed - may need to be readjusted if
% the document is modified later
%\IEEEtriggeratref{8}
% The "triggered" command can be changed if desired:
%\IEEEtriggercmd{\enlargethispage{-5in}}

% references section

% can use a bibliography generated by BibTeX as a .bbl file
% BibTeX documentation can be easily obtained at:
% http://www.ctan.org/tex-archive/biblio/bibtex/contrib/doc/
% The IEEEtran BibTeX style support page is at:
% http://www.michaelshell.org/tex/ieeetran/bibtex/
%\bibliographystyle{IEEEtran}
% argument is your BibTeX string definitions and bibliography database(s)
%\bibliography{IEEEabrv,../bib/paper}
%
% <OR> manually copy in the resultant .bbl file
% set second argument of \begin to the number of references
% (used to reserve space for the reference number labels box)
\begin{thebibliography}{1}

\bibitem{mit}Lec 1 | MIT 6.00 Introduction to Computer Science and Programming, Fall 2008, \url{https://www.youtube.com/watch?v=k6U-i4gXkLM}, Aug 19, 2009.

\bibitem{harvard}Lecture 0 - Introduction to Computer Science I, \url{https://www.youtube.com/watch?v=z-OxzIC6pic}, May 11, 2014.

\bibitem{peking}Professor Daniel Hassidim in MBA lecture at Peking University - part 2, \url{https://www.youtube.com/watch?v=yX8NBrVEHwU}, Jun 10, 2011.

\bibitem{hongkong}KEEP Seminar at The Hong Kong University of Science and Technology, \url{https://www.youtube.com/watch?v=oatUyBaXzbA}, Jan 12, 2016.

\bibitem{delft}TU Delft - Professor Walter Lewin: Rainbows and Blue Skies, \url{https://www.youtube.com/watch?v=eDByEKf2IEM}, Jun 13, 2013.

\bibitem{amsterdam}Alan Fenwick – first part of Lecture at University of Amsterdam, \url{https://www.youtube.com/watch?v=5hMH-ExBwM4}, Jun 25, 2016.

\bibitem{hildebrand1971evaluating} 
Hildebrand, Milton and others.
\textit{Evaluating University Teaching.}
ERIC, 1971.

\end{thebibliography}

% biography section
% 
% If you have an EPS/PDF photo (graphicx package needed) extra braces are
% needed around the contents of the optional argument to biography to prevent
% the LaTeX parser from getting confused when it sees the complicated
% \includegraphics command within an optional argument. (You could create
% your own custom macro containing the \includegraphics command to make things
% simpler here.)
%\begin{biography}[{\includegraphics[width=1in,height=1.25in,clip,keepaspectratio]{mshell}}]{Michael Shell}
% or if you just want to reserve a space for a photo:


% You can push biographies down or up by placing
% a \vfill before or after them. The appropriate
% use of \vfill depends on what kind of text is
% on the last page and whether or not the columns
% are being equalized.

%\vfill

% Can be used to pull up biographies so that the bottom of the last one
% is flush with the other column.
%\enlargethispage{-5in}



% that's all folks
\end{document}


